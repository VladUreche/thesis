\begingroup
\let\cleardoublepage\clearpage

% English abstract
\cleardoublepage
\chapter*{Abstract}
%\markboth{Abstract}{Abstract}
\addcontentsline{toc}{chapter}{Abstract} % adds an entry to the table of contents

\vspace{0.7em}

High-level languages allow programmers to express data structures and algorithms that abstract over the type of data they handle. This improves code reuse and makes it possible to develop general-purpose libraries. Yet, data abstractions slow down program execution, as they require low-level indirection. In this thesis we explore three compile-time approaches that leverage type systems to reduce the cost of data abstractions, thus improving program performance.

\vspace{0.7em}

In the first part of the thesis we present miniboxing, a compile-time transformation that replaces generic classes by more efficient variants, optimized to handle primitive types. These variants use the miniboxed data encoding, producing speedups of up to 20$\times$ compared to generic classes. The miniboxing transformation is the main result of this thesis and motivates the other techniques.

\vspace{0.7em}

Generalizing miniboxing, we show the Late Data Layout (LDL) mechanism, which uses the type system to guide performance-oriented program rewritings. It can be instantiated to perform a host of transformations, such as miniboxing generics, inlining value classes and unboxing primitive types. The LDL mechanism has many desirable properties, such as provable correctness in handling different data representations, reduced number of conversions and built-in support for the object-oriented paradigm.

\vspace{0.7em}

Finally, we show Data-centric Metaprogramming, a technique that allows programmers to go beyond standard compiler optimizations by defining custom representations to be used for their data. These representations are then automatically introduced by the compiler when translating programs. This technique, similar in spirit to metaprogramming, opens new directions in programmer-driven optimizations and shows encouraging results, with speedups of up to 25$\times$. Under the hood, Data-centric Metaprogramming uses the Late Data Layout mechanism.

\vspace{0.7em}

% Finally, we show how all the techniques above collaborate to make the miniboxing transformation optimally handle the entire Scala programming language, including constructs such as functional interfaces, tuples and type classes, which are automatically transformed using a customized version of the Data-centric Metaprogramming technique. We also explain the principles used by the miniboxing plugin to guide programmers to improve their code through actionable performance advisories.

% \vspace{1.0em}
Key words:
Data Representation; Transformation; Object-Oriented; Static Type System; Performance; Generics; Specialization; Java; Java Virtual Machine; Bytecode; Semantics.




% % German abstract
% \begin{otherlanguage}{german}
% \cleardoublepage
% \chapter*{Zusammenfassung}
% %\markboth{Zusammenfassung}{Zusammenfassung}
% % put your text here
% \lipsum[1-2]
% \vskip0.5cm
% Stichwörter:
% %put your text here
% \end{otherlanguage}




% French abstract
\begin{otherlanguage}{french}
\cleardoublepage
\chapter*{Résumé}
%\markboth{Résumé}{Résumé}

\vspace{0.5em}

Les langages de programmation haut niveau permettent aux programmeurs de développer des structures de données et algorithmes en faisant abstraction du type de données qu'ils gèrent. Cela permet la réutilisation du code et le développement des bibliothèques d'usage général. Mais l'abstraction de données a pourtant un coût sur la performance, en raison de plusieurs indirections bas-niveau que cette dernière introduit. Cette thèse explore trois techniques de compilation qui utilisent les systèmes de typage afin d'améliorer les performances d’un programme en réduisant le coût des abstractions de données.

\vspace{0.3em}

Dans la première partie de la thèse, nous présentons le “miniboxing”, une transformation qui remplace des classes génériques par des variantes optimisées pour gérer les types primitifs. Ces variantes utilisent un encodage “miniboxé” des données, produisant des programmes qui peuvent être jusqu'à 20$\times$ plus rapides. Le “miniboxing” est le résultat principal de la thèse et motive les autres techniques.

\vspace{0.3em}

En généralisant le miniboxing, nous introduisons le mécanisme de “Late Data Layout” (LDL), qui utilise le système de typage pour guider les transformations. Il peut être utilisé pour effectuer une multitude de optimisations, tels que le miniboxing, l’”inlining” des “value classes”, et l’élimination du “boxing” des types primitifs. Le mécanisme de LDL a de nombreuses propriétés souhaitables: nous prouvons que les différentes manipulations de données sont correctes, que le nombre de conversions entre différents formats est minimisé, et nous intégrons cette transformation dans les langages orienté objet.

\vspace{0.3em}

Finalement, nous introduisons la métaprogrammation centrée sur les données (Data-centric Metaprogramming), une technique qui permet aux programmeurs d'aller au-delà des optimisations standard du compilateur, et de définir des représentations sur mesure pour leur données. Ces représentations sont ensuite automatiquement utilisées par le compilateur lors de la transformation d'un programme. Cette technique, dans le même esprit que la métaprogrammation, ouvre de nouvelles directions dans les optimisations dirigées par le programmeur et montre des résultats encourageants, produisant des programmes jusqu'à 25$\times$ plus rapides. A la base, la “Data-centric Metaprogramming” utilise le mécanisme “Late Data Layout”.

\vspace{0.3em}

% Finally, we show how all the techniques presented collaborate to make the miniboxing transformation optimally handle the entire Scala programming language, including constructs such as functional interfaces, tuples and type classes, which are automatically transformed using a customized version of the Data-centric Metaprogramming technique. We also explain the principles used by the miniboxing plugin to guide programmers to improve their code through actionable performance advisories.

% \vspace{1.0em}
Mots clefs:

Représentation des données; Transformation; Programmation orientée objet; Systèmes de typage statiques; Performance; Generics; Spécialisation; Java; Java Virtual Machine; Bytecode; Sémantique.
\end{otherlanguage}


%\endgroup
%\vfill
